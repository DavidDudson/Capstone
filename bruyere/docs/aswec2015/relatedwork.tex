\section{Related Work}

Several authors have reported the positive impact gamification can have on educational outcomes \cite{muratet2009towards,rankin2008impact,wang2009application}. The PrimeGame was first used in 2003 at the University of Applied Sciences Gelsenkirchen and since 2009 at the Polytechnic of Namibia and some other institutions in Germany \cite{meyer2010primegame}. Since then, successful annual and bi-annual competitions took place, with good participation rates, although the overall success-rate depended on the cultural background of students and required careful balancing between the competitive and collaborative aspects of the game. In \cite{meyer2011primegame}, the authors presented a version of the PrimeGame based on a client-server architecture but using a proprietary Java client.

There is large body of research on educational programming environments. For space reasons, we only discuss the projects that have directly inspired our work. Greenfoot \cite{henriksen2004greenfoot} is a desktop-based development environment where students produce animations by programming actors in Java. Greenfoot can be seen as a turn-based one-player game environment. It supports some social interaction via an online gallery. GreenFoot's focus is on teaching OOP concepts, following the ``objects first'' philosophy. We have used GreenFoot successfully for years in a 200-level object-oriented programming course at Massey University. Robocode \cite{nelson2001robocode} is a turn-based multiplayer game. While robocode is client-based, it supports collaboration through battle simulators that can load robots developed by other players.  Scratch \cite{resnick2009scratch} is an online environment for creating simple animations and games. It uses its own visual language, and targets a younger audience. For a more detailed comparison between scratch, greenfoot and the related alice environment, the reader is referred to \cite{utting2010alice}.

There are a number of web-based development environments that have inspired our work, in particular the cloud9 IDE\footnote{\url{https://c9.io/}} and the Python Tutor \cite{GuoSIGCSE2013}. 

%A unique feature of the Python tutor is the program execution visualisation.
