\section{User Interface}

Figures \ref{fig:editor} -- \ref{fig:test} show the browser-based user interface. The editor (Figure \ref{fig:editor}) provides a simple web-based IDE based on the widely used ACE component\footnote{\url{http://ace.c9.io/}}. The number of available functions has been minimised to retain the
simplicity necessary for educational programming environments. The editor supports syntax highlighting and formatting. 
Support for auto-completion is planned but not yet available. 

Note the sharing feature in the main menu. This flags the bot as shared, and produces a URL that can be shared on social networks to invite other users to play against this bot. 

\begin{figure}
	\includegraphics[width=8.9cm]{figures/editor.pdf}
	\caption{SoGaCo code editor}
	\label{fig:editor}
\end{figure}

Bots are submitted to the server for storing and building. As discussed above, the server processes the bot in a build pipeline. Errors that occur at the various stages are displayed to the user in the console panel underneath the code editor. Also, a marker is set to highlight the critical code causing the problem. This includes compilation errors. In addition, violations for the several verification rules are displayed. An example can be seen in Figure \ref{fig:verification}. Here the user has attempted to store a bot that forces the JVM to exit (\texttt{System.exit(0)}). This is discovered during the static bytecode verification step (BYTE\_CHECK),  the respective error message is displayed and the line of code is highlighted.


\begin{figure}
	\centering
	\includegraphics[width=8.9cm]{figures/verification.pdf}
	\caption{SoGaCo code editor reporting a verification error}
	\label{fig:verification}
\end{figure}

Games can be played in the test environment shown in Figure \ref{fig:test}. While the code editor is generic, the test environment is game specific. The games are executed on the server, the results are recorded, (JSON-) encoded and returned to the client for animated replay. The animation controls are on the right side of the screen. The left side contains a list of bots, two bots must be selected from the list to play a game. These bots are colour-coded upon selection (red and blue, respectively). Note that bots shared by other users can be looked up and selected as well. When a shared bot URL is loaded, the test page is loaded with the shared bot preselected as opponent.

In the game shown in Figure \ref{fig:test}, a simple cautious bot plays against a more sophisticated (blue) bot that plays the largest prime number available. 

\begin{figure}
	\centering
	\includegraphics[width=8.9cm]{figures/test.pdf}
	\caption{SoGaCo PrimeGame test environment}
	\label{fig:test}
\end{figure}

