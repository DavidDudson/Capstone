\section{Introduction}


There are widely acknowledged shortcomings in computer science education that is often seen as unable to satisfy the needs of employers, both in terms of the quality and the quantity of the graduates produced. The use of gamification seems to be a viable approach to improve the experience students have, and can improve student engagement and educational outcomes \cite{muratet2009towards,rankin2008impact,wang2009application}.

On the other hand, educators face additional challenges: budgets to acquire new systems are tied, and there is a general expectation that programming tools can be made available for free or at a very low cost. Moreover, teachers at highschool level have very limited technical support and often no access to servers to deploy custom educational packages. This leads in many cases to a preference for cloud-based systems that have a low overall total costs of ownership.

In this paper,
we describe our experience in building such a platform. We focus on the technical problems we had to solve to achieve a sufficient level of usability, stability, security and scalability. The system described is SoGaCo (social gaming and coding). SoGaCo is a platform that supports content modules build around simple mathematical board games. Students write bots that play those games on their behalf, and share those bots to play against peers or assessment / benchmarking bots.

While we have implemented several different games with SoGaCo\footnote{PrimeGame, Mancala, Othello, 5 in a Row}, our discussion focuses on our experience with one particular game, the PrimeGame \cite{meyer2010primegame,meyer2011primegame}. We chose the PrimeGame for the following reasons: (1) The PrimeGame has an extremely simple programming model that makes it suitable for entry level programming for both highschool and first year university level teaching. (2) There is a simple hierarchy of possible strategies with increasing complexity and increasing bot strength. This increase requires students to explore new features of the programming language used (conditionals, loops, container data structures). (3) The PrimeGame has been used successfully in tertiary education before.


The rest of the paper is organised as follows: we review related work first, followed by a detailed discussion of the design, security and scalability aspects and a brief presentation of the game user interface. A public installation of the game is available at \url{http://sogaco.massey.ac.nz/}.